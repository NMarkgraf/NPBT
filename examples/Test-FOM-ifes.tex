\PassOptionsToPackage{unicode=true}{hyperref} % options for packages loaded elsewhere
\PassOptionsToPackage{hyphens}{url}
\PassOptionsToPackage{dvipsnames,svgnames*,x11names*}{xcolor}
%
\documentclass[10pt,ngerman,a4paper,ignorenonframetext,]{beamer}
\usepackage{pgfpages}
\setbeamertemplate{caption}[numbered]
\setbeamertemplate{caption label separator}{: }
\setbeamercolor{caption name}{fg=normal text.fg}
\beamertemplatenavigationsymbolsempty
\usepackage{lmodern}
\usepackage{amssymb,amsmath}
\usepackage{ifxetex,ifluatex}
\usepackage{fixltx2e} % provides \textsubscript
\ifnum 0\ifxetex 1\fi\ifluatex 1\fi=0 % if pdftex
  \usepackage[T1]{fontenc}
  \usepackage[utf8]{inputenc}
  \usepackage{textcomp} % provides euro and other symbols
\else % if luatex or xelatex
  \usepackage{unicode-math}
  \defaultfontfeatures{Ligatures=TeX,Scale=MatchLowercase}
\fi
\usetheme[]{NPBT}
\usecolortheme{NPBT-FOM-ifes}
% use upquote if available, for straight quotes in verbatim environments
\IfFileExists{upquote.sty}{\usepackage{upquote}}{}
% use microtype if available
\IfFileExists{microtype.sty}{%
\usepackage[]{microtype}
\UseMicrotypeSet[protrusion]{basicmath} % disable protrusion for tt fonts
}{}
\IfFileExists{parskip.sty}{%
\usepackage{parskip}
}{% else
\setlength{\parindent}{0pt}
\setlength{\parskip}{6pt plus 2pt minus 1pt}
}
\usepackage{xcolor}
\usepackage{hyperref}
\hypersetup{
            pdftitle={Norman's Pandoc Beamer Themes},
            pdfauthor={Norman Markgraf},
            colorlinks=true,
            linkcolor=darkgray,
            citecolor=Blue,
            urlcolor=blue,
            breaklinks=true}
\urlstyle{same}  % don't use monospace font for urls
\newif\ifbibliography
% Prevent slide breaks in the middle of a paragraph:
\widowpenalties 1 10000
\raggedbottom
\setbeamertemplate{part page}{
\centering
\begin{beamercolorbox}[sep=16pt,center]{part title}
  \usebeamerfont{part title}\insertpart\par
\end{beamercolorbox}
}
\setbeamertemplate{section page}{
\centering
\begin{beamercolorbox}[sep=12pt,center]{part title}
  \usebeamerfont{section title}\insertsection\par
\end{beamercolorbox}
}
\setbeamertemplate{subsection page}{
\centering
\begin{beamercolorbox}[sep=8pt,center]{part title}
  \usebeamerfont{subsection title}\insertsubsection\par
\end{beamercolorbox}
}
\AtBeginPart{
  \frame{\partpage}
}
\AtBeginSection{
  \ifbibliography
  \else
    \frame{\sectionpage}
  \fi
}
\AtBeginSubsection{
  \frame{\subsectionpage}
}
\setlength{\emergencystretch}{3em}  % prevent overfull lines
\providecommand{\tightlist}{%
  \setlength{\itemsep}{0pt}\setlength{\parskip}{0pt}}
\setcounter{secnumdepth}{0}

% set default figure placement to htbp
\makeatletter
\def\fps@figure{htbp}
\makeatother

\ifnum 0\ifxetex 1\fi\ifluatex 1\fi=0 % if pdftex
  \usepackage[shorthands=off,main=ngerman]{babel}
\else
  % load polyglossia as late as possible as it *could* call bidi if RTL lang (e.g. Hebrew or Arabic)
  \usepackage{polyglossia}
  \setmainlanguage[]{german}
\fi

\title{Norman's Pandoc Beamer Themes}
\author{Norman Markgraf}
\providecommand{\institute}[1]{}
\institute{Sefiroth Consulting}
\date{06.02.2017}

\begin{document}
\frame{\titlepage}

\hypertarget{ein-theme-fur-verschiedene-anwendungen}{%
\section{Ein Theme für verschiedene
Anwendungen}\label{ein-theme-fur-verschiedene-anwendungen}}

\begin{frame}[fragile]{Wozu dieses Theme}
\protect\hypertarget{wozu-dieses-theme}{}

Dieses Theme entstand um mit einem Basis-Text schnell Präsentationen für
verschiedene Bereiche zu erstellen.

Für ein einfaches FOM Layout muss unter

\begin{verbatim}
- -V
- theme=NPBT
\end{verbatim}

Die Zeile

\begin{verbatim}
- -V
- colortheme=NPBT-FOM
\end{verbatim}

eingefügt werden.

Statt \texttt{NPBT-FOM} gibt es auch \texttt{NPBT-EUFOM} und
\texttt{NPBT-FOM-ifes} als mögliche Farbthemen.

\end{frame}

\begin{frame}{Wie geht's es weiter?}
\protect\hypertarget{wie-gehts-es-weiter}{}

Ich versuche noch ein wenig das allgemeine FOM Layout nachzuarmen. Aber
ich denke für den schnellen Einsatz ist es so schon gut genug. Der Kern
geht diese Woche auf GitHub. Die ``Module'' für die FOM etc. gibt es
dann ggf. auf Anfrage ;-)

\end{frame}

\hypertarget{ein-theme-fur-verschiedene-anwendungen-1}{%
\section{Ein Theme für verschiedene
Anwendungen}\label{ein-theme-fur-verschiedene-anwendungen-1}}

\begin{frame}[fragile]{Wozu dieses Theme}
\protect\hypertarget{wozu-dieses-theme-1}{}

Dieses Theme entstand um mit einem Basis-Text schnell Präsentationen für
verschiedene Bereiche zu erstellen.

Für ein einfaches FOM Layout muss unter

\begin{verbatim}
- -V
- theme=NPBT
\end{verbatim}

Die Zeile

\begin{verbatim}
- -V
- colortheme=FOM
\end{verbatim}

eingefügt werden.

Statt FOM gibt es auch EUFOM und FOM\_ifes als mögliche Formate.

\end{frame}

\begin{frame}{Wie geht's es weiter?}
\protect\hypertarget{wie-gehts-es-weiter-1}{}

Ich versuche noch ein wenig das allgemeine FOM Layout nachzuarmen. Aber
ich denke für den schnellen Einsatz ist es so schon gut genug. Der Kern
geht diese Woche auf GitHub. Die ``Module'' für die FOM etc. gibt es
dann ggf. auf Anfrage ;-)

\end{frame}

\end{document}
